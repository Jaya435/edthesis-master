\chapter{Conclusion}
The objective of this project was to create an easily reproducible and implementable convolutional neural network, specifically for remote sensing problems. The work presented here and in \citet{Richmond19b}, provide a framework within which a remote sensing scientist would be able to train and test on their data and evaluate the results. This paper also suggests methods to be implemented in the next iteration to improve the results. 
\par
While the results of the training are secondary to the workflow, they are not as significant as was originally assumed. Although the model was successfully trained, an optimal set of hyperparameters was not found in the time available and as a result, the accuracy was low.
\par
Overall, the project was a success. Convolutional neural networks are a deep learning technique that has a large amount of potential for the remote sensing community. While there are already successful land classification CNNs, this project has tried to simplify the process so that scientists with access to smaller datasets are still able to implement their CNNs for an area of interest and a single, specific land-use class.